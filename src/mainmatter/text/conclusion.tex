\chapter{Conclusion}

% A one- or two-paragraph account of what the take-away of the entire
% presentation experience has been for you. The focus here could be general
% reflection on the whole process – from connecting one’s core purpose to the
% why of the presentation, from team brainstorming, individual research and
% language acquisition to rehearsal, the use of presentation
% techniques/language, and the delivery of prosodic elements. The focus of the
% conclusion can also hone in on any one of these aspects before it extrapolates
% to a more general statement on the take-away experience.

% The conclusion can be seen as a phenomenological reflection on the subjective
% learning experience across all phases and challenges of the presentation
% process and may thus be written in the first person. You may quote material
% from the reader or another source.

% I'm quite passionate about the topic of this presentation, as I've been struggling on and off for close to a year now to fully grasp it.
I'm quite passionate about the topic of this presentation, as I've had trouble understanding the concept myself for close to year now.
Being able to not just understand it myself, but to be able to convey the information in a hopefully clear and concise manner, represents a somewhat important achievement to me.
The need for this study assignment gave the motivation to finally finish up on this topic.

The presentation itself was somewhat different as to what I would have done normally.
As the presentation is very technical, I would usually simplify the concepts massively.
However, as my audience was knowledgeable in the field, I felt I could get away with diving into the interesting but complex technical details.
The added structure provided by the signposting, \textit{VACOG} system, etc. was certainly helpful.
Although the concepts themselves weren't particularly new to me, they provide a more formal insight into not just how to make a presentation interesting but into why these techniques work.
I've definitely enjoyed the process of creating and holding this presentation.

\asterism
