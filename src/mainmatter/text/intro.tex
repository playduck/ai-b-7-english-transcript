\chapter{Introduction}

% The introduction should state the study assignment’s purpose and give an
% overview of its parts. The purpose here could be presenting the outcomes of a
% project, for example.

This study assignment contains a transcript and analysis of my presentation titled \enquote{Psychoacoustic Noise-Shaping}.
It was held on the \DTMdisplaydate{2022}{12}{21}{} on the premises of the \textit{Deggendorf Institute of Technology (DIT)}.

The presentation intended to introduce the audience to the topic of psychoacoustic noise-shaping.
The main goal was to argue for the necessity of the technique, explain its concepts and mechanism and furthermore clarify all steps to a certain point.
The goal was not to enable the listener to be able to implement such an algorithm but rather to introduce and demystify the topic in an approachable and understandable manner.
The topic was chosen as it contains relatively simple operations of digital signal processing which are applied to a real, rather unknown, and nearly-impossible sounding application.

This paper does not focus on the topic of noise-shaping.
It's simply reciting parts of a transcript and offers an analysis on the presentation itself -- not its content.
