\subsection{Analysis}

% One page analysis of your transcript in which you give a detailed breakdown of
% the presentation language and techniques/elements you have used and why you
% have used them. Here you quote your own transcript and give an analysis of the
% communicational efficacy of using this language and/or these
% techniques/elements. You can also quote material from the reader or another
% source. Please use full sentences in this part and make sure to distinguish
% your analysis from the presentation content you are analysing, e.g., by using
% quotation marks. Any language that is not directly relevant to your analysis
% should be marked with an ellipsis.

The presentation starts of in a very nonchalant manner.
Addressing the audience to set up expectations of a technical and difficult topic and referencing the previously held presentations.
There is no grand introduction or \textit{Bang}.
However, this is not in violation of what \citeauthor{powell2011presenting} says with \enquote{Don't waste time on long boring introductions.}\autocite[Point 3 on Page 7]{powell2011presenting}.
As the audience's attention is captured in two ways:
Partially from the ominous subtitle of the presentation (\enquote{Exploiting biology in digital audio}) which isn't addressed directly -- An interesting and somewhat intriguing claim.
The other part is simply the contrast of the casual appearance with virtually all other presenters up to that point.
This is again used in reference to \citeauthor{powell2011presenting}s Point 7: \enquote{speak naturally} \autocite[Page 7]{powell2011presenting}.
It creates a more relaxed atmosphere at the start.
Which nicely flows into the signposting, which relies heavily on the rule of three \autocite[Page 59, A]{williams2008presentations}.
This is one of the few parts in the presentation where I'm relying on dreaded bullet points:
\enquote{[\ldots] \enquote{bullets}--little black circles in front of phrases that were supposed to summarize things.
There was one after another of these little goddamn bullets [\ldots] on the slides.}\autocite[Page 126--127]{Feynman}.
The slides feature very little text and bullet points to focus the attention to the presenter and only aid in displaying visuals.

This relaxed atmosphere is then put into stark contrast once the introduction and signposting is over.
Demanding the audiences' attention as I introduce a fairly common scenario vividly: Converting a digital signal to an analog signal.
By moving around and guiding the attention I can introduce the depicted situation in an understandable and controlled way without relying on animations, but still keeping it interesting.
As this is probably the least technical aspect I felt comfortable to include a question to the audience:
\enquote{How do you go from the digital domain to the analog domain? Any ideas?}
This introduces the main problem of the topic: Conversion from a high Bit-Depth to a lower one -- Quantization.
This problem is relevant in the real world as this scenario is truly ubiquitous.
The relevance is again underlined by mentioning how the application for this is found \enquote{\ldots in every phone, laptop, whatever plays back audio really}.
At this point the audience knows what the problem and the context is: Quantization and signal processing respectively.

The presentation goes on with explaining the effects this quantization creates.
As visually explaining the differences is immensely more simple, the presentation relies heavily on graphs and diagrams.
The third excerpt is in reference to a set of graphs.
% All graphs are exaggerated to show the point I want to make.
% They're not meant to be scientific displays of real data.
These are not derived from real world data but rather exaggerated to clearly demonstrate the point I want to make.
As such the focus is on readability and ease of understanding.
Again referencing \citeauthor{powell2011presenting} with Point 13: \enquote{Let your visual speak for themselves}\autocite[Page 8]{powell2011presenting}.
However, I didn't want to leave the audience guessing for themselves what a given chart is trying to say.
For every graph I gave a quick overview to orient the listener:
\enquote{On the X-Axis we have our frequency -- not numbered, this is just an example. The Y-Axis, again, represents the amplitude.}

The midpoint of the presentation starts after explaining the dilemma: We either have noise or distortion, which both are undesirable.
The question arises if we can't just do something else to avoid this situation.
To which the answer is no, we can't break the laws of physics.
The revolutionary revelation of not removing the noise, but simply \enquote{push{[\textit{ing}]} the noise somewhere else} is then revealed though a cartoon interlude.
The characters in the altered video each offer ridiculous solutions: A comedic break from the technical topics, which makes strong uses of the visual and auditory senses as defined by the \textit{VAKOG} system in \autocite[Page 77]{williams2008presentations}.
This simple concept is then re-stated in a more formal way in the shape of a graph, again using \textit{VAKOG}.
For easy reference it's again depicted in \autoref{fig:noise-shaping-spectrum}.
Again a quick introduction is given with:
\enquote{This green area shows the area of the signal we're interested in. The red one shows the noise.}
An interesting detail here is the use of color.
Throughout the presentation the same colors have been used to refer to conceptually similar ideas
(Red always being noise and green always being the quantized signal\footnote{The colors are taken directly from the \textit{DIT} branding guidelines and are used by the faculties \textcolor{THDGreen}{\textit{BIW}}, \textcolor{THDDarkRed}{\textit{AWW}} and \textcolor{THDStrongBlue}{\textit{MB-MK}} giving the presentation a homogenous and harmonious look.}).

The ending of the presentation consists of a quick re-cap by going over some key figures.
This re-iterates the journey and all logical steps taken to get to the end result.
The slides then end on the obligatory sources.
However, as time was still plenty, I decided not to end the presentation here and instead show a concrete example of the topic in action.
Showing the source code and running it to see the effect in action on real data further fundaments the understanding as well as demystifies the complex concepts.
It wasn't originally planned, but worked out nonetheless, as it acts as a closing \textit{Bang}, as mentioned by \citeauthor{williams2008presentations} on Page 30.
This, again, creates a more natural and relaxed environment:
It's less of a lecture and more of a conversation.
