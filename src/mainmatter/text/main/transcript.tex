\subsection{Transcript}

% PART I: Half a page transcript of a representative part of your segment of the
% presentation. At least half of your transcript should be devoted to the
% technical part of your presentation. The other half can focus on the opening,
% close, or another part of the presentation’s main body. Please note that
% transcript means that you re-produce in writing what you produced or would
% like to have produced in the real-time presentation.

All technical sources from the presentation are from
\autocites{frequency-sampling-method}{noise-shaping}{shannon}{dither}{iso226}{SASPWEB2011}{frequency-sampling-method-2}{dsp-guide}.

\enquote{Alright, so just as a preliminary remark:
    This presentation's going to be a \textit{bit} more technical than the last few.
    So you're probably not going to understand everything right away -- so I'm going to refer you to the handout on \textit{iLearn}, where you can find additional information and sources on the topic ---
    Anyway, today I'm going to talk to you about \textit{psychoacoustic noise-shaping}.
    I'd usually start off by asking if anyone knows what this is, but quite frankly you probably don't, so we'll just skip that.
    Before jumping in, let's cover what we'll cover in this presentation \ldots
}

\asterism

\enquote{So why do we need all this?
    Well, picture the following scenario:
    You have a digital audio source over here.
    That could be your smartphone, a computer, Bluetooth, whatever.
    And on the other side you have an analog audio sink: So headphones or loudspeakers.
    But now there's a slight problem:
    How do you go from the digital domain to the analog domain? Any ideas?
    \begin{verse}
        \enquote{%
        How about a digital to analog converter?}
    \end{verse}
    Correct we need a DAC (digital to analog converter).
    Now this setup is pretty ubiquitous.
    Same thing's in every phone, laptop, whatever plays back audio really.
    Now here's the catch however:
    DACs need data in a fixed-point format, often something like 16-Bits fixed.
    However, we like doing digital audio processing on the computer in floating point.
    Usually something like 32-Bits float.
    So now we need to convert from float to fixed.
}

\asterism

\enquote{Let's take a look at these signals discrete Fourier-Transform in the frequency domain\footnote{This is in reference to \autoref{fig:orig} in the Appendix.}.
    On the X-Axis we have our frequency -- not numbered, this is just an example.
    The Y-Axis, again, represents the amplitude.
    The first signal, the 32-Bit one, looks pretty nice:
    There are some artifacts of the transformation up here, but we can ignore those.
    We have this nice bump in here, which represents our fundamental frequency -- the sine wave.
    Let's take a look at the quantized version\footnote{In reference to \autoref{fig:quant} in the Appendix.}.
    It looks good up to the fundamental, but then -- what is all this mess up here?
    This is the distortion which introduces harmonic overtones.
}

\asterism

\vspace{8pt}
\begin{verse}
    \enquote{\texttt{\ldots What if we take the noise and \textit{push} it somewhere else? --- }}\\
    \enquote{\texttt{That idea might just be crazy enough\ldots}} \footnote{
        Text in \texttt{monospace} is audio from a video interlude used to reveal the key idea of noise-shaping.}\\
\end{verse}
\enquote{%
    \ldots to actually work.
    So picture this, X-Axis, again, the frequency and Y the amplitude \footnote{This is in reference to \autoref{fig:noise-shaping-spectrum} in the Appendix}.
    This green area shows the area of the signal we're interested in.
    The red one shows the noise.
    So what if instead of having this uniform noise floor, where there's equal amounts of noise at every frequency, we just \textit{push} the noise somewhere else.
    Now note that we didn't add or remove any of the noise; It's still the same.
    It's just that the area of noise intersecting our signal is greatly reduced, thus essentially decreasing the noise in our signal.
    This is the entire concept of noise-shaping.
    Now how does this actually work \ldots
}
